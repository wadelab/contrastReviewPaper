% Options for packages loaded elsewhere
\PassOptionsToPackage{unicode}{hyperref}
\PassOptionsToPackage{hyphens}{url}
\PassOptionsToPackage{dvipsnames,svgnames,x11names}{xcolor}
%
\documentclass[
  letterpaper,
  DIV=11,
  numbers=noendperiod]{scrartcl}

\usepackage{amsmath,amssymb}
\usepackage{iftex}
\ifPDFTeX
  \usepackage[T1]{fontenc}
  \usepackage[utf8]{inputenc}
  \usepackage{textcomp} % provide euro and other symbols
\else % if luatex or xetex
  \usepackage{unicode-math}
  \defaultfontfeatures{Scale=MatchLowercase}
  \defaultfontfeatures[\rmfamily]{Ligatures=TeX,Scale=1}
\fi
\usepackage{lmodern}
\ifPDFTeX\else  
    % xetex/luatex font selection
\fi
% Use upquote if available, for straight quotes in verbatim environments
\IfFileExists{upquote.sty}{\usepackage{upquote}}{}
\IfFileExists{microtype.sty}{% use microtype if available
  \usepackage[]{microtype}
  \UseMicrotypeSet[protrusion]{basicmath} % disable protrusion for tt fonts
}{}
\makeatletter
\@ifundefined{KOMAClassName}{% if non-KOMA class
  \IfFileExists{parskip.sty}{%
    \usepackage{parskip}
  }{% else
    \setlength{\parindent}{0pt}
    \setlength{\parskip}{6pt plus 2pt minus 1pt}}
}{% if KOMA class
  \KOMAoptions{parskip=half}}
\makeatother
\usepackage{xcolor}
\setlength{\emergencystretch}{3em} % prevent overfull lines
\setcounter{secnumdepth}{-\maxdimen} % remove section numbering
% Make \paragraph and \subparagraph free-standing
\ifx\paragraph\undefined\else
  \let\oldparagraph\paragraph
  \renewcommand{\paragraph}[1]{\oldparagraph{#1}\mbox{}}
\fi
\ifx\subparagraph\undefined\else
  \let\oldsubparagraph\subparagraph
  \renewcommand{\subparagraph}[1]{\oldsubparagraph{#1}\mbox{}}
\fi


\providecommand{\tightlist}{%
  \setlength{\itemsep}{0pt}\setlength{\parskip}{0pt}}\usepackage{longtable,booktabs,array}
\usepackage{calc} % for calculating minipage widths
% Correct order of tables after \paragraph or \subparagraph
\usepackage{etoolbox}
\makeatletter
\patchcmd\longtable{\par}{\if@noskipsec\mbox{}\fi\par}{}{}
\makeatother
% Allow footnotes in longtable head/foot
\IfFileExists{footnotehyper.sty}{\usepackage{footnotehyper}}{\usepackage{footnote}}
\makesavenoteenv{longtable}
\usepackage{graphicx}
\makeatletter
\def\maxwidth{\ifdim\Gin@nat@width>\linewidth\linewidth\else\Gin@nat@width\fi}
\def\maxheight{\ifdim\Gin@nat@height>\textheight\textheight\else\Gin@nat@height\fi}
\makeatother
% Scale images if necessary, so that they will not overflow the page
% margins by default, and it is still possible to overwrite the defaults
% using explicit options in \includegraphics[width, height, ...]{}
\setkeys{Gin}{width=\maxwidth,height=\maxheight,keepaspectratio}
% Set default figure placement to htbp
\makeatletter
\def\fps@figure{htbp}
\makeatother
% definitions for citeproc citations
\NewDocumentCommand\citeproctext{}{}
\NewDocumentCommand\citeproc{mm}{%
  \begingroup\def\citeproctext{#2}\cite{#1}\endgroup}
\makeatletter
 % allow citations to break across lines
 \let\@cite@ofmt\@firstofone
 % avoid brackets around text for \cite:
 \def\@biblabel#1{}
 \def\@cite#1#2{{#1\if@tempswa , #2\fi}}
\makeatother
\newlength{\cslhangindent}
\setlength{\cslhangindent}{1.5em}
\newlength{\csllabelwidth}
\setlength{\csllabelwidth}{3em}
\newenvironment{CSLReferences}[2] % #1 hanging-indent, #2 entry-spacing
 {\begin{list}{}{%
  \setlength{\itemindent}{0pt}
  \setlength{\leftmargin}{0pt}
  \setlength{\parsep}{0pt}
  % turn on hanging indent if param 1 is 1
  \ifodd #1
   \setlength{\leftmargin}{\cslhangindent}
   \setlength{\itemindent}{-1\cslhangindent}
  \fi
  % set entry spacing
  \setlength{\itemsep}{#2\baselineskip}}}
 {\end{list}}
\usepackage{calc}
\newcommand{\CSLBlock}[1]{\hfill\break\parbox[t]{\linewidth}{\strut\ignorespaces#1\strut}}
\newcommand{\CSLLeftMargin}[1]{\parbox[t]{\csllabelwidth}{\strut#1\strut}}
\newcommand{\CSLRightInline}[1]{\parbox[t]{\linewidth - \csllabelwidth}{\strut#1\strut}}
\newcommand{\CSLIndent}[1]{\hspace{\cslhangindent}#1}

\KOMAoption{captions}{tableheading}
\makeatletter
\@ifpackageloaded{caption}{}{\usepackage{caption}}
\AtBeginDocument{%
\ifdefined\contentsname
  \renewcommand*\contentsname{Table of contents}
\else
  \newcommand\contentsname{Table of contents}
\fi
\ifdefined\listfigurename
  \renewcommand*\listfigurename{List of Figures}
\else
  \newcommand\listfigurename{List of Figures}
\fi
\ifdefined\listtablename
  \renewcommand*\listtablename{List of Tables}
\else
  \newcommand\listtablename{List of Tables}
\fi
\ifdefined\figurename
  \renewcommand*\figurename{Figure}
\else
  \newcommand\figurename{Figure}
\fi
\ifdefined\tablename
  \renewcommand*\tablename{Table}
\else
  \newcommand\tablename{Table}
\fi
}
\@ifpackageloaded{float}{}{\usepackage{float}}
\floatstyle{ruled}
\@ifundefined{c@chapter}{\newfloat{codelisting}{h}{lop}}{\newfloat{codelisting}{h}{lop}[chapter]}
\floatname{codelisting}{Listing}
\newcommand*\listoflistings{\listof{codelisting}{List of Listings}}
\makeatother
\makeatletter
\makeatother
\makeatletter
\@ifpackageloaded{caption}{}{\usepackage{caption}}
\@ifpackageloaded{subcaption}{}{\usepackage{subcaption}}
\makeatother
\ifLuaTeX
  \usepackage{selnolig}  % disable illegal ligatures
\fi
\usepackage{bookmark}

\IfFileExists{xurl.sty}{\usepackage{xurl}}{} % add URL line breaks if available
\urlstyle{same} % disable monospaced font for URLs
\hypersetup{
  pdftitle={Measuring contrast processing in the visual system using SSVEP},
  pdfauthor={Alex R. Wade; Daniel H. Baker},
  colorlinks=true,
  linkcolor={blue},
  filecolor={Maroon},
  citecolor={Blue},
  urlcolor={Blue},
  pdfcreator={LaTeX via pandoc}}

\title{Measuring contrast processing in the visual system using SSVEP}
\author{Alex R. Wade \and Daniel H. Baker}
\date{}

\begin{document}
\maketitle

Department of Psychology and York Biomedical Research Institute,
University of York, UK

\subsection{Review for Vis Neurosci}\label{review-for-vis-neurosci}

\section{Abstract:}\label{abstract}

Contrast is the currency of the early visual system. Measuring the way
that the computations underlying contrast processing depend on factors
such as spatial and temporal frequency, age, eccentricity, chromaticity
and the presence of other stimuli has been a focus of vision science for
at least 100 years. One of the most productive experimental approaches
in this field has been the use of the `steady-state visually-evoked
potential' (SSVEP): a technique where contrast modulating inputs are
`frequency tagged' (presented at well-defined frequencies and phases)
and the electrical signals that they generate in the brain are analyzed
in the temporal frequency domain. SSVEPs have several advantages over
conventional measures of visually-evoked responses: they have relatively
unambiguous ouput measures, a high SNR, and they allow us to analyze
interactions between stimulus components using a convenient mathematical
framework. Here we describe how SSVEPs have been used to study visual
contrast over the past 70 years (Dawson, 1954). Because our thinking
about SSVEPs is well-described by simple mathematical models, we embed
code that illustrates key steps in the modelling and analysis. This
paper can therefore be used both as a review of the use of SSVEP in
measuring human contrast processing, and as an interactive learning aid.

\section{Introduction}\label{introduction}

Neurons in the visual areas of the brain are primarily responsive to
changes in cone photoreceptor activations across time and space. This
property, referred to as contrast, sets the fundamental limits of our
visual abilities, which remain steady over a remarkably wide range of
environmental light levels. The human response to contrast can be
studied using many different techniques. Early work used psychophysical
methods to measure contrast sensitivity (Campbell \& Green, 1965),
defined as the inverse of the lowest contrast that can be reliably
detected. But neural responses can also be measured more directly using
techniques such as fMRI, MEG, and EEG. Here we will describe how an EEG
method known as the steady state visually evoked potential (SSVEP)
technique has contributed to our understanding of human contrast
processing in health, disease and throughout development.

The SSVEP is a continuous electrical response evoked in the brain by
visual stimuli flickering at a constant frequency (Regan, 1966). For
contrast-defined stimuli, such as sine-wave gratings, it is strongest at
the occipital pole, adjacent to the early visual areas that generate the
signal, although careful analysis of individual VEPs reveals multiple
generators throughout visual cortex (Di Russo \emph{et al.}, 2005,
2007). The flickering stimulus entrains neural population responses at
multiples of the stimulus frequency, so continuous EEG data are
typically analysed by taking the Fourier transform, and estimating the
amplitude at these frequencies. Two common variants involve sinusoidal
on-off flicker, where the stimulus alternates between a blank background
and the peak contrast, and sinusoidal counterphase flicker, where the
stimulus alternates in phase (i.e the black regions become white and the
white regions become black). On-off flicker can drive independent
populations of on- and off-cells in the retina once per cycle and can
therefore produce a response at the fundamental flicker frequency, known
as 1F, and its integer harmonics: 2F, 3F, 4F and so on. Counterphase
flicker contains two transients per cycle and therefore does not produce
a response at 1F, only at its even harmonics: 2F, 4F, 6F and so on.
Because square-waves are spectrally broad-band, square wave flicker
tends to produce more complex spectral harmonics than sine-wave flicker.

The higher harmonics of the steady-state signal are generally thought to
reflect nonlinear processing in the visual system (Regan \& Regan,
1988). SSVEP signals can also be elicited by periodic changes of
stimulus properties other than achromatic and chromatic contrast, such
as motion, stereo depth, and facial identity or expression (see Norcia
\emph{et al.}, 2015, for an overview); however our focus here is on the
contrast response.

\section{Why measure responses to
contrast?}\label{why-measure-responses-to-contrast}

Contrast is one of the most fundamental pieces of information that the
eye transmits to the brain. It can be defined as the change in cone
photoreceptor activity over space (`spatial contrast') or time
(`temporal contrast'). Cone photoreceptors - which drive precortical
opponent pathways - contribute to both chromatic and achromatic contrast
and although most of the research we describe here focuses on achromatic
contrast, SSVEPs have proven to be an excellent measure of early
chromatic processing as well (McKeefry \emph{et al.}, 1996; Baseler \&
Sutter, 1997; Di Russo \emph{et al.}, 2001).

Contrast is relatively simple to define: typically, contrast is
specified as the percentage deviation of a uniform stimulus from the
background. So, for example, a disk of 100 units of cone activation
(\(I_{\mathrm{stim}}\)) surrounded by a `background' of 50 units of
activation (\(I_{\mathrm{background}}\)) has a contrast of
\(\frac{I_{\mathrm{stim}} - I_{\mathrm{background}}}{I_{\mathrm{background}}}\)
= 100\%. Where patterns are more complex (for example, the sine-wave
gratings or Gabor patches common in vision science), the Michelson
(1927) definition of contrast is specified by the maximum and minimum
excursions from the mean:

\begin{equation}
\frac {I_{\mathrm{stimmax} }-I_{\mathrm{stimmin} }}{I_{\mathrm{stimmax} }+I_{\mathrm{stimmin} }}.
\end{equation}

These contrast definitions are appropriate both to photometric measures
of stimulus contrast (for example, luminance; Lennie \emph{et al.}
(1993)) and also to definitions based on cone excitations (MacLeod \&
Boynton, 1979; Derrington \emph{et al.}, 1984) which are more common in
work on chromatic processing.

Although its mathematical definition is straightforward, the
computations that underlie contrast processing in the brain have been
the subject of intense research for many decades. The neural code for
contrast, even in the earliest parts of visual cortex, is not simply a
linear transform of the contrast at the retina - instead, contrast
signals undergo a cascade of nonlinear processing stages that, broadly,
attempt to normalise the output relative to the spatiotemporal
environment. This normalization, achieved through a computation called
`contrast gain control' (Heeger, 1992; Foley, 1994; Carandini \& Heeger,
2011) maximises the sensitivity of the visual system by making optimal
use of neuronal bandwidth. As an example, a grating placed at the centre
of a low-contrast background typically appears more intense than the
same grating when superimposed on a high contrast background (see
Figure~\ref{fig-centresurround}).

\begin{figure}

\centering{

\includegraphics{review_files/figure-pdf/fig-centresurround-output-1.pdf}

}

\caption{\label{fig-centresurround}The perceived contrast of a stimulus
depends on its context. A high contrast surround reduces the apparent
contrast of the central `probe' region''}

\end{figure}%

A significant body of research into contrast processing is concerned
with how these normalization mechanisms depend on colour (Chen \emph{et
al.}, 2000), orientation (Foley, 1994), eye of origin (Legge, 1979;
Baker \emph{et al.}, 2007), spatial and temporal frequency (Meese \&
Baker, 2009), location (Polat \& Sagi, 1993; Tadin \emph{et al.}, 2003;
Petrov \emph{et al.}, 2005), age (Betts \emph{et al.}, 2005), and the
presence of neurological disorders (Porciatti \emph{et al.}, 2000; Tsai
\emph{et al.}, 2011). The SSVEP has proven to be invaluable in this
research because it provides an objective readout of contrast
representation at different stages of the visual system, and allows us
to `tag' the probe and background at separate frequencies.

Because it provides a direct read-out of neural population activity, the
SSVEP signal can reveal key features of neural signal transduction. For
example, by varying the peak stimulus contrast parametrically, a
`contrast vs response' function can be measured - where the `response'
is typically defined as the amplitude of the SSVEP frequency component
at the stimulus frequency or a low multiple thereof. This corresponds
closely to similar functions reported by studies measuring single unit
activity or local field potentials in the cortex (Shapley \& Victor,
1980; Morrone \emph{et al.}, 1982). However the SSVEP has the advantage
that it is non-invasive, and so can be measured in awake, behaving human
participants.

To understand the utility of the contrast SSVEP, it is helpful to
identify the cascade of processing stages in the early visual system
that give rise to it. In the following section we illustrate how a
typical SSVEP signal measured over early visual cortex might contain
information about a large number of early visual computations.

\section{Contrast processing - linear and
nonlinear}\label{contrast-processing---linear-and-nonlinear}

Neurons have a limited dynamic range, yet they can transmit information
about visual stimuli that span many orders of magnitude. In the domain
of contrast, to some extent this is accomplished at a population level -
individual neurons typically implement a non-linear, sigmoidal contrast
vs response (CRF) transducer (Tolhurst \emph{et al.}, 1981; Albrecht \&
Hamilton, 1982) and different neurons exhibit peak sensitivity (defined
as the maximum slope of the function) at different contrast levels
(Carandini \& Heeger, 1994; Carandini \emph{et al.}, 1998; Busse
\emph{et al.}, 2009). A neuronal population will therefore span a
sensitivity range greater than any individual member.

Individual neurons at multiple stages of the visual hierarchy also
change their sensitivity depending on the average spatiotemporal
contrast energy of their environment. This ``normalisation'' process is
dynamic and nonlinear and is well-modeled by a hyperbolic ratio function
in which the response of each neuron is modulated by a local `gain pool'
composed of the summed responses of the local neuronal population
(Heeger, 1992; Busse \emph{et al.}, 2009; Carandini \& Heeger, 2011;
Baker \& Wade, 2017).

Previous figures showed how a single-input SSVEP might generate a single
set of peaks in the frequency domain response which, in turn, can be
used as a read-out of neuronal activity. In
Figure~\ref{fig-plotspectra}, we show the effect of adding a second
contrast component: the two components interact generating nonlinear
intermodulation terms at sums and differences of the input frequencies
(F1, F2). The nature of the interaction --- and therefore the pattern of
intermodulation terms --- is determined by the computations happening as
the contrast signal moves from retina to cortex (Regan \& Regan, 1988)

First, we implement a simple non-linear transducer with a single input
to illustrate the typical sigmoidal shape of a contrast response
function (Figure~\ref{fig-contrastresponsefunction}).

Next, we show that adding a second component which contributes to the
gain pool of the first, results in a complex pattern of intermodulation
terms (Figure~\ref{fig-plotspectra}).

\begin{figure}

\centering{

\includegraphics{review_files/figure-pdf/fig-plotsinewaves-output-1.pdf}

}

\caption{\label{fig-plotsinewaves}a) A typical SSVEP setup: a screen, a
grating, a flicker rate: b) The responses from a single electrode (time
domain) c) Frequency domain}

\end{figure}%

The two inputs are represented here as amplitude modulations across
time.

If the two inputs are simply added together, the representation of the
resulting signal in the Fourier domain is just the linear sum of the two
independent signals (first two panels of Figure~\ref{fig-plotspectra}).
However, it is important to take physiology into account. If we ignore
cells sensitive to contrast transients, contrast is represented as a
combination of rectified signals from `on' and `off'-sensitive cells
(broadly, cells that code positive or negative changes in spatiotemporal
luminance). To model the population response of these cells to a
contrast-reversing luminance sine wave, we therefore introduce an
initial full-wave rectification stage that simulates the outputs of two,
equal populations of on- and off-polarity detectors. Once this is done,
we see that the Fourier spectrum of the resulting signal contains power
only at the second harmonics and above (final two panels of
Figure~\ref{fig-plotspectra}). It is possible to generate first-hamonic
responses in contrast-response SSVEPs, but only when the inputs are
modulated in an `on/off' manner rather than contrast reversed.

\begin{figure}

\centering{

\includegraphics{review_files/figure-pdf/fig-plotspectra-output-1.pdf}

}

\caption{\label{fig-plotspectra}Temporal frequency representations of
combining two inputs at frequencies F1=5Hz and F2=7Hz combination. A
pure sine-wave input would generate responses only at F1 and F2.
However, the rectification that accompanies early visual processing
converts this contrast reversal into a more complex frequency-domain
representation with power predominantly at the second harmonics 2F1 and
2F2}

\end{figure}%

Even without considering a spatial component, the early visual system is
far more complex than the model here suggests. For example as well as
cells that code positive or negative contrast in a more or less
continuous manner, the retina also contains `transient' cells that code
temporal changes in contrast. These cells (Kuffler \emph{et al.}, 1957;
Alpern, 1971), and cells with similar properties in the LGN (Levitt
\emph{et al.}, 2001) and cortex (Hubel \& Wiesel, 1959; Movshon, 1975),
will introduce second harmonic components even when the stimulus itself
is modulated in an on-off fashion. Analogously, in the spatial domain,
so-called `simple cells' are sensitive to the polarity of a spatial
contrast modulation while `complex cells' respond to the presence of
patterned spatial contrast irrespective of its spatial position. The
SSVEP response to a contrast-reversing sine-wave grating therefore
contains information about nonlinear computations performed across a
range of retinal and cortical cell types.

\begin{figure}

\centering{

\includegraphics{review_files/figure-pdf/fig-contrastresponsefunction-output-1.pdf}

}

\caption{\label{fig-contrastresponsefunction}Non-linear contrast
response function resulting from the population average of two full-wave
rectified signals passing through a non-linear transducer. Note the
`supersaturation' at high contrast levels.}

\end{figure}%

One of the simplest nonlinearities is the function that describes a
cell's response to different levels of contrast. In
Figure~\ref{fig-contrastresponsefunction} this is modeled as described
above by a hyperbolic ratio function resulting in a saturating
non-linearity:

\begin{equation}
R_{\mathrm{max}} = \frac{C_{\mathrm{in}}^n}{C_{50}^n + C_{\mathrm{in}}^n},
\end{equation}

where \({R_{\mathrm{max}}}\) describes the maximum response level,
\(C_{\mathrm{in}}\) is the input contrast, \(C_{50}\) is the
`semi-saturation constant' (the point at which the response is at
half-maximum) and \emph{n} controls the steepness of the curve.

The hyperbolic ratio function is monotonic but the CRF resulting from
measuring the amplitude of the second harmonic (2F) component contains a
slight roll-off at high input contrasts. This results from the
distortion of the input sine waves at high contrast due to a combination
of the full-wave rectification and saturating non-linearity. Power at
other harmonics increases, and the total power generated by the input is
monotonic. This roll-off is often seen in experimental data and has been
referred to as `supersaturation' (Tyler \& Apkarian, 1985; Peirce,
2007).

To illustrate the effect of contrast gain control (Heeger, 1992), we
include a second component that contributes to the gain pool of the
first. At a single pair of (matched) contrast levels, we see a complex
pattern of intermodulation terms in the FT
(Figure~\ref{fig-intermodulation}).

\begin{figure}

\centering{

\includegraphics{review_files/figure-pdf/fig-intermodulation-output-1.pdf}

}

\caption{\label{fig-intermodulation}Non-linear combination of two
independent contrast modulations at F1 and F2 results in
`intermodulation' terms that appear at low-order sums and differences of
those frequencies}

\end{figure}%

The complexity of even a simple simulation of the frequency-domain
signal arising from non-linear interactions is intriguing. Presumably,
the signal measured from early visual cortex contains is the result of a
cascade of nonlinear retinal and cortical operations up to that point.
It therefore contains a `signature' of the computational nature, order
and parameters of those operations - including the shape of the
transducer functions and the computations involved in signal
combination. In principle, that information could be recovered from the
SSVEP signal.

This possibility was recognised in the early days of the technique
(Regan \& Regan, 1988). Although characterising the complete set of
computations along entire processing pathway is problematic, careful
parametric variation of the input stimuli does allow us to fit models of
early visual gain control using SSVEP data. \ldots{} MORE HERE -
Although this idea was suggested as early as 1988 (probably earlier) not
many people seem to have done it much. Candy \emph{et al.} (2001); Busse
\emph{et al.} (2009); Tsai \emph{et al.} (2012); Baker \& Wade (2017)
*** SEE ALSO RECENT WORK FROM WINAWER LAB - MUCH OF IT IS `FAST' VEPs
***

\section{Measuring the development of contrast
processing}\label{measuring-the-development-of-contrast-processing}

An early use of the SSVEP was to provide an objective estimate of
spatial contrast sensitivity in infants, without requiring behavioural
responses. In well-motivated adults, psychophysical measurements of
contrast sensitivity remain the gold standard. However, it is difficult
and time consuming to obtain reliable psychophysical data from infants.
In these cases, SSVEP measurements represent a fast and efficient method
for measuring low-level visual responses (Tyler \emph{et al.}, 1979;
Braddick \emph{et al.}, 1986; Norcia \emph{et al.}, 1990) and the high
SNR of SSVEP means that infants need only look at the screen for short
periods of time.

Because SSVEP responses at detection threshold are very small,
estimating a threshold is achieved by measuring the contrast response
function at relatively high levels, and extrapolating back along the
function (either contrast vs response measured at a constant spatial
frequency or spatial frequency vs response at a constant contrast level)
to estimate its intercept with the x-axis (see
Figure~\ref{fig-sweepvep}). This contrast level was shown to correspond
approximately with psychophysically measured detection thresholds
(Norcia \emph{et al.}, 1986).

\begin{figure}

\centering{

\includegraphics{review_files/figure-pdf/fig-sweepvep-output-1.pdf}

}

\caption{\label{fig-sweepvep}Sweep VEP simulation showing how a contrast
detection threshold can be estimated from sweep VEP data measured at
higher contrasts}

\end{figure}%

A robust estimate of the threshold therefore requires the measurement of
the SSVEP amplitude at many different super-threshold contrasr levels.
This was made faster by the development of the `sweep VEP' paradigm in
which the stimulus changed its contrast, spatial frequency or some other
property, throughout a trial (Tyler \emph{et al.}, 1979). To avoid
hysteresis effects, the sweep is sometimes conducted both up and down in
the same experiment Norcia \& Tyler (1985\emph{a}). The sweep VEP
(really, a sweep `SSVEP') technique is now commonly-used to obtain a
rapid and objective measurement of visual acuity. In particular, because
of its relative speed and simplicity, this technique has now become a
standard for conducting tests of visual acuity in very young subjects or
where behavioural tests are not appropriate (Ridder, 2004; Bach \emph{et
al.}, 2008; Hoffmann \emph{et al.}, 2017; Bach \& Farmer, 2020).

This approach has revealed much about the development of visual
abilities in infants (Harris \emph{et al.}, 1976; Atkinson \emph{et
al.}, 1979; Braddick \emph{et al.}, 1986). In general, SSVEP
measurements of infant vision have revealed that contrast sensitivity
for both achromatic and chromatic contrast as well as stereoscopic depth
perception develops earlier than had been supposed previously based on
behavioural readouts (Dobson \emph{et al.}, 1978; Norcia \& Tyler,
1985\emph{b}) with both chromatic and achromatic contrast detection
reaching near-adult levels by around six months. Spatial acuity as
measured by SSVEP reaches adult levels more slowly, but near-adult
levels are recorded around one year (Norcia \& Tyler, 1985\emph{b};
Hamer \emph{et al.}, 1989) compared to around six to seven years with
behavioural measures (Atkinson \& Braddick, 1983; Ellemberg \emph{et
al.}, 1999). At least some of this difference is likely due to the
relative objectivity and high SNR of the SSVEP technique compared to
other methods such as preferential looking, which require careful
measurement of the infant's gaze direction. However it should be noted
that other groups have reported electrophysiological correlates of
visual acuity that more closely match the behavioural measures (De
Vries-Khoe \& Spekreijse, 1982).

The SSVEP technique has also been used to study the development of the
contrast gain control mechanisms described in the previous section
(Candy \emph{et al.}, 2001; Pei \emph{et al.}, 2017). Although contrast
gain control is measurable in infants as young as six weeks old (Morrone
\& Burr, 1986; Skoczenski \& Norcia, 1998; Candy \emph{et al.}, 2001),
its development appears to be slower, with adult levels being reached at
approximately 11 years (Pei \emph{et al.}, 2017).

\section{Contrast processing in clinical
conditions}\label{contrast-processing-in-clinical-conditions}

The SSVEP technique has also been used to study clinical conditions,
such as diseases and developmental disorders. This can often be
informative regarding the underlying mechanism that characterises the
condition. Here we focus on four conditions, but there is potential to
apply the method more broadly, either as a diagnostic technique, or to
monitor disease severity and progression, or the efficacy of treatments.

Epilepsy is a neurological condition in which patients experience
seizures - episodes of uncontrolled neural activity that can cause
unconsciousness, involuntary movements and convulsions, and atypical
sensory experiences. Porciatti \emph{et al.} (2000) showed that
individuals with photosensitive epilepsy generate larger steady-state
signals in response to flickering visual stimuli, and their contrast
response functions saturate less than those of healthy controls. This is
consistent with the idea that epilepsy involves a cortical
hyperexciteability that makes seizures more likely. It is also the case
for individuals with idiopathic generalised epilepsy (Tsai \emph{et
al.}, 2011), a subtype of epilepsy that has a less obvious link to
vision. The differences apply across the whole contrast-response
function, and so resemble a response gain effect (see
Figure~\ref{fig-plotclinical}a), which might be due to reduced
inhibition from neighbouring neurons. Differences in SSVEP amplitudes
have also been reported in individuals with migraine (Shibata \emph{et
al.}, 2008), a condition also associated with cortical
hyperexciteability.

Amblyopia is a disorder of binocular vision, characterised by one eye
contributing much less to perception than the other. This is often due
to strabismus (squint) or anisometropia (difference in optical
prescription between the eyes) during development. Contemporary accounts
suggest that the amblyopic eye is suppressed by signals from the fellow
eye. SSVEPs provide a convenient and objective method to characterise
the difference in neural response to signals in each eye, and typically
show reduced responses to stimuli in the amblyopic eye across the
contrast range (Baker \emph{et al.}, 2015; Lygo \emph{et al.}, 2021).
There are currently many novel binocular treatments for amblyopia under
development, often involving virtual reality or stereo display systems
designed to encourage the two eyes to work together. The steady-state
approach may be more sensitive and objective than typical acuity
measurements, and also has the potential to measure suppression between
the eyes directly (e.g. Zheng \emph{et al.}, 2019; Hu \emph{et al.},
2023; Du \emph{et al.}, 2023).

Autism is a condition often associated with differences in vision
(Simmons \emph{et al.}, 2009) and other senses (MacLennan \emph{et al.},
2022). Pei \emph{et al.} (2014) used a sweep-VEP method with
counterphase flickering stimuli, and found weaker responses in autistic
children at spatial frequencies around 8c/deg, compared with age-matched
controls. This was subsequently replicated in a further pediatric sample
by Vilidaite \emph{et al.} (2018) (see Figure~\ref{fig-plotclinical}c),
who additionally found weaker responses in autistic adults at the second
harmonic (using on/off flicker). Interestingly this study replicated its
key findings in a \emph{Drosophila} genetic model of autism (\emph{Nhe3}
mutations), illustrating the translational potential of the steady-state
approach, as well as identifying a possible biomarker for autism.

Parkinson's

Alex - it would be good to add some CRFs from Afsari \emph{et al.}
(2014) to the figure. Is it possible to extract either the fitted
parameter values or the raw data for some control flies and some
mutants? Whatever you think is most compelling.

\begin{figure}

\centering{

\includegraphics{review_files/figure-pdf/fig-plotclinical-output-1.pdf}

}

\caption{\label{fig-plotclinical}Example contrast response functions for
different clinical conditions. Panel (a) shows modelled contrast
response functions for epilepsy patients (blue) and control participants
(black), based on the data of Tsai et al.~(2011). Panel (b) shows
functions for the amblyopic (red) and fellow (black) eyes of adults with
amblyopia, based on the data of Baker et al.~(2015). Panel (c) shows
functions for children with (green) and without (black) a diagnosis of
autism, based on the data of Vilidaite et al.~(2018). Panel (d)}

\end{figure}%

\section{Brain-computer interfaces}\label{brain-computer-interfaces}

\section{Conclusions}\label{conclusions}

\emph{2: Sweeps and CRFs. Measurements of contrast sensitivity,
extrapolating the sweep to zero response to get t'hold. Infants and
adults?}

\emph{3: Measurements of modulation. Figures adapted from other papers:
Attention to space, attention to features, adaptation (?).
masking/surround suppression\textbackslash{}}

\emph{4: Clinical (Porciatti / Tsai/ Marmite/ Amblyopia / PD?)}

\emph{5: Future directions (decoding in frequency domain?, animals?
BCI?)}

\phantomsection\label{anchor-2}{}History

\phantomsection\label{anchor-3}{}1. The basics of SSVEP and contrast
sensitivity including a history of both fields

SSVEP (Steady-State Visual Evoked Potential): A continuous electrical
response evoked in the brain by visual stimuli flickering at a constant
frequency (Regan, 1966).

Contrast Sensitivity: The ability to detect differences in luminance
between an object and its background (Campbell \& Green, 1965).

Regan, D. (1966). Some characteristics of average steady-state and
transient responses evoked by modulated light. Electroencephalography
and clinical neurophysiology, 20(3), 238-248.

Campbell, F. W., \& Green, D. G. (1965). Optical and retinal factors
affecting visual resolution. J Physiology, 181, 576-593.

Check Regan paper for earlier (e.g.~EEG refs). Norcia review will be
helpful!

(From Tyler / Levi / Apkarian paper):

6. Regan, D.: Rapid methods for refracting the eye and assessing the
visual acuity in amblyopia using steady-state visual evoked potentials.
In Desmedt, J.E., editor: Visual Evoked Potentials in Man: New
Developments, Oxford, 1977, Clarendon Press, pp.~418-426.

7. Fricker, S.J.: Narrow-band filter techniques for the detection and
measurement of evoked responses, Electroencephalogr. Clin. Neurophysiol.
14:411, 1962.

8. Van der Tweel, L. H., Sem-Jacobsen, C.W., Kamp, A., Van Leeuwen,
W.S., and Verings, F.T.H.: Objective determination of response to
modulated light, Acta Physiol. Pharmacol. Neerl. 7:528, 1958.

9. Regan, D.: Latencies of evoked potentials to flicker and to pattern
speedily estimated by simultaneous stimulation method,
Electroencephalogr. Clin. Neurophysiol. 40:654, 1976.

10. Tyler, C.W., Apkarian, P., and Nakayama, K.: Multiple spatial
frequency tuning of electrical responses from the human.\ldots{}

{[}Around here perhaps a section about what gain control is, mentioning
other methods as well including psychophysics, MRI, electrophysiology,
and other EEG markers including ERPs and evoked gamma band oscillations.
Maybe outline the Heeger gain control model and its cousins.{]}

{[}Sure. But in fact gain control is only a small part of this story -
especially in the early days. They were, I think, more interested in
using SSVEP to measure absolute contrast sensitivity and to get lower
bounds on things like infant visual development. Gain control might be
better as a separate section later.{]}

\phantomsection\label{anchor-4}{}2. Spatial, temporal frequency and
contrast sensitivity measurements

Basically using SSVEP to measure a cortical output amplitude for any
given input contrast. You can vary parameters like SF, TF, position,
color and of course contrast. Early on people realised that you can
`sweep' the stimulus to get a CRF. You broadly get a line in log
contrast space if you do that (Tyler) - then you can extrapolate that
line down to zero response to estimate the threshold. That doesn't
\textgreater quite\textless{} work but it's pretty close.

You can also use this for `difficult' populations like babies. One
interesting story was about how SSVEP become a replacement for
preferential looking (which was the other way of looking at infant
visual development). See e.g.~Davida Teller. SSVEPs allowed people to
make objective measurements of contrast sensitivity development and
deduce that the visual system was more mature (e.g.~more functional) in
infancy than previously expected. Also measures of colour sensitivity.
Tinyeyes is based off those measurements. Other people: Tyler, Norcia,
Gunilla H-P, many of the people at SKERI in the 1980s and 90s. Norcia
86,88,90 - mentioned in Regan's nice autobiography :

In parallel of course, people were using frequency tagging to do single
unit work - the 1F vs 2F simple/complex cell classification scheme was
all about this (Lennie and others).

\phantomsection\label{anchor-5}{}3. SSVEP in functional localization

Techniques like fMRI have been combined with SSVEP to achieve more
precise spatial localization (Di Russo et al., 2007). See however the
Ales cruciform paper.

Ales and Norcia:... (showed that people's intuition about V1 upper /
lower v.f reversing polarity wasn't really correct). This feels a little
outside our scope though... We have also combined SSVEP with source
imaging techniques to probe responses in different cortical locations
(again, Ales papers, some early stuff from Stan K? Appelbaum, Wade,
Norcia figure/ground...) and then a host of later work from that lab and
others.

\phantomsection\label{anchor-6}{}4. SSVEP and contrast gain control
including adaptation, masking, and attention

{[}The story goes: Up to the lte 90s people were primarily interested in
measuring contrast sensitivity - the shape of the response function was
assumed to be basically log-linear - and they fit it with straight lines
to extrapolate back to zero response. But then (once Heeger's 1992 paper
had sunk in - see also stuff like Shapley and Victor 1981
\href{https://www.zotero.org/google-docs/?zAmtgj}{(Shapley \& Victor,
1981)}), people started thinking about gain control - Candy and Norcia
in about 1999, Porciatti, probably a load of Tyler papers that I don't
even know about...{]}. Probably look in p

And then people worked out that if you can use SSVEP to measure contrast
responses, you can also use it to measure things that modulate contrast
responses. These include adaptation, masking, suppression, attention
(feature and space), clinical things.

\phantomsection\label{anchor-7}{}Adaptation:

Continuous exposure to high-contrast patterns reduces contrast
sensitivity, which can be measured using SSVEP (Ross et al., 1989).
Others? Baker recent gain control paper is perhaps worth mentioning here
as a `confound' of sorts. Engel 2018
\href{https://www.zotero.org/google-docs/?kdURiS}{\emph{(Vergeer et al.,
2018)}}. This paper is interesting
\href{https://www.zotero.org/google-docs/?J2EO3d}{\emph{(Rideaux et al.,
2023)}} and \textgreater sort\textless{} of SSVEP.

\phantomsection\label{anchor-8}{}Masking:

High contrast masks can suppress the visibility of low contrast
patterns, which has implications in SSVEP amplitude (Haynes et al.,
2003).

Attention: Directing attention can enhance contrast sensitivity, as
shown in studies using SSVEP (Müller et al., 2006). Also Tsai
(dynamics), Baker / Wade (several), Winawer? I think JW has a nice
dynamic model of normalization with some MEG data. Busse et al cat/human
comparison. Candy and Norcia 2001 JNS
\href{https://www.zotero.org/google-docs/?7hhs5Z}{(Candy et al., 2001)}

Ross, J., Speed, H. D., \& Morgan, M. J. (1989). The effects of
adaptation and masking on incremental thresholds for contrast. Vision
research, 29(2), 205-215.

Haynes, J. D., Roth, G., Stadler, M., \& Heinze, H. J. (2003).
Neuromagnetic correlates of perceived contrast in primary visual cortex.
Journal of Neurophysiology, 89(6), 2655-2666.

Müller, M. M., Picton, T. W., Valdes-Sosa, P., Riera, J.,
Teder-Sälejärvi, W. A., \& Hillyard, S. A. (2006). Effects of spatial
selective attention on the steady-state visual evoked potential in the
20--28 Hz range. Cognitive Brain Research, 24(1), 1-13.

5. Clinical implications

\phantomsection\label{anchor-9}{}Clinical applications (this is a whole
section)

Not sure exactly how SSVEP used in clinic. mfVEP?

, e.g., in monitoring visual impairments, tracking neuronal diseases, or
neurofeedback (Norcia et al., 2015).

Tsai epilepsy \href{https://www.zotero.org/google-docs/?AdhH3f}{(Tsai et
al., 2011)}. Other photosensitive epilepsy: Porciatti
\href{https://www.zotero.org/google-docs/?UTBQbV}{(Porciatti et al.,
2000)}

Citation:

Migraine (Regan et al). Autism?

Norcia, A. M., Appelbaum, L. G., Ales, J. M., Cottereau, B. R., \&
Rossion, B. (2015). The steady-state visual evoked potential in vision
research: A review. Journal of vision, 15(6), 4-4.

Animal work: Flies (Eliott, West, Himmelberg, Ales, Norcia): Mice/rats
(probably many - can't think off the top of my head - we were working
with a mouse EEG person at UCSF about 12 years ago...), Monkeys
(Kiorpes?),

\phantomsection\label{anchor-10}{}6. Future directions

Use as a readout of modulations. TMS? FUS? Marmite B12 / Fluoxetine /
amblyopia in Rats,

GABA Huang

There is a lot of SSVEP interest these days because of BCIs. I think
it's pretty weak but there is \textgreater so much\textless{} of it that
it might be worth mentioning...

Advanced signal processing techniques and machine learning can be
integrated to improve SSVEP-based systems (Zhu et al., 2010).

Exploring new clinical and diagnostic applications, understanding
neurological diseases, and developing novel therapeutic interventions.

Citation:

Zhu, D., Bieger, J., Molina, G. G., \& Aarts, R. M. (2010). A survey of
stimulation methods used in SSVEP-based BCIs. Computational intelligence
and neuroscience, 2010.

\section*{References}\label{references}
\addcontentsline{toc}{section}{References}

\phantomsection\label{refs}
\begin{CSLReferences}{1}{1}
\bibitem[\citeproctext]{ref-Afsari2014}
Afsari F, Christensen KV, Smith GP, Hentzer M, Nippe OM, Elliott CJH \&
Wade AR (2014). \href{https://doi.org/10.1093/hmg/ddu159}{Abnormal
visual gain control in a {Parkinson}'s disease model}. \emph{Human
Molecular Genetics} \textbf{23,} 4465--4478.

\bibitem[\citeproctext]{ref-Albrecht1982}
Albrecht DG \& Hamilton DB (1982).
\href{https://doi.org/10.1152/jn.1982.48.1.217}{Striate cortex of monkey
and cat: Contrast response function}. \emph{J Neurophysiol} \textbf{48,}
217--237.

\bibitem[\citeproctext]{ref-Alpern1971}
Alpern M (1971).
\href{https://doi.org/10.1113/jphysiol.1971.sp009580}{Rhodopsin kinetics
in the human eye}. \emph{J Physiol} \textbf{217,} 447--471.

\bibitem[\citeproctext]{ref-Atkinson1983}
Atkinson J \& Braddick O (1983).
\href{https://doi.org/10.1111/j.1755-3768.1983.tb03927.x}{Assessment of
visual acuity in infancy and early childhood}. \emph{Acta Ophthalmol
Suppl} \textbf{157,} 18--26.

\bibitem[\citeproctext]{ref-Atkinson1979}
Atkinson J, Braddick O \& French J (1979).
\href{https://www.ncbi.nlm.nih.gov/pubmed/761974}{Contrast sensitivity
of the human neonate measured by the visual evoked potential}.
\emph{Invest Ophthalmol Vis Sci} \textbf{18,} 210--213.

\bibitem[\citeproctext]{ref-Bach2020}
Bach M \& Farmer JD (2020).
\href{https://doi.org/10.1007/s10633-019-09726-2}{Evaluation of the
"freiburg acuity VEP" on commercial equipment}. \emph{Doc Ophthalmol}
\textbf{140,} 139--145.

\bibitem[\citeproctext]{ref-Bach2008}
Bach M, Maurer JP \& Wolf ME (2008).
\href{https://doi.org/10.1136/bjo.2007.130245}{Visual evoked
potential-based acuity assessment in normal vision, artificially
degraded vision, and in patients}. \emph{Br J Ophthalmol} \textbf{92,}
396--403.

\bibitem[\citeproctext]{ref-Baker2007}
Baker DH, Meese TS \& Summers RJ (2007).
\href{https://doi.org/10.1016/j.neuroscience.2007.01.030}{Psychophysical
evidence for two routes to suppression before binocular summation of
signals in human vision}. \emph{Neuroscience} \textbf{146,} 435--448.

\bibitem[\citeproctext]{ref-Baker2015}
Baker DH, Simard M, Saint-Amour D \& Hess RF (2015).
\href{https://doi.org/10.1167/iovs.14-15611}{Steady-state contrast
response functions provide a sensitive and objective index of amblyopic
deficits}. \emph{Invest Ophthalmol Vis Sci} \textbf{56,} 1208--1216.

\bibitem[\citeproctext]{ref-Baker2017}
Baker DH \& Wade AR (2017).
\href{https://doi.org/10.1093/cercor/bhw395}{Evidence for an {Optimal}
{Algorithm} {Underlying} {Signal} {Combination} in {Human} {Visual}
{Cortex}}. \emph{Cerebral Cortex (New York, NY: 1991)} \textbf{27,}
254--264.

\bibitem[\citeproctext]{ref-Baseler1997}
Baseler HA \& Sutter EE (1997).
\href{https://doi.org/10.1016/s0042-6989(96)00209-x}{M and p components
of the VEP and their visual field distribution}. \emph{Vision Res}
\textbf{37,} 675--690.

\bibitem[\citeproctext]{ref-Betts2005}
Betts LR, Taylor CP, Sekuler AB \& Bennett PJ (2005).
\href{https://doi.org/10.1016/j.neuron.2004.12.041}{Aging reduces
center-surround antagonism in visual motion processing}. \emph{Neuron}
\textbf{45,} 361--366.

\bibitem[\citeproctext]{ref-Braddick1986}
Braddick OJ, Wattam-Bell J \& Atkinson J (1986).
\href{https://doi.org/10.1038/320617a0}{Orientation-specific cortical
responses develop in early infancy}. \emph{Nature} \textbf{320,}
617--619.

\bibitem[\citeproctext]{ref-Busse2009}
Busse L, Wade AR \& Carandini M (2009).
\href{https://doi.org/10.1016/j.neuron.2009.11.004}{Representation of
concurrent stimuli by population activity in visual cortex}.
\emph{Neuron} \textbf{64,} 931--942.

\bibitem[\citeproctext]{ref-Campbell1965}
Campbell FW \& Green DG (1965).
\href{https://doi.org/10.1113/jphysiol.1965.sp007784}{Optical and
retinal factors affecting visual resolution}. \emph{J Physiol}
\textbf{181,} 576--593.

\bibitem[\citeproctext]{ref-Candy2001}
Candy TR, Skoczenski AM \& Norcia AM (2001).
\href{https://doi.org/10.1523/JNEUROSCI.21-12-04530.2001}{Normalization
models applied to orientation masking in the human infant}. \emph{The
Journal of Neuroscience: The Official Journal of the Society for
Neuroscience} \textbf{21,} 4530--4541.

\bibitem[\citeproctext]{ref-Carandini1994}
Carandini M \& Heeger DJ (1994).
\href{https://doi.org/10.1126/science.8191289}{Summation and division by
neurons in primate visual cortex}. \emph{Science} \textbf{264,}
1333--1336.

\bibitem[\citeproctext]{ref-Carandini2011}
Carandini M \& Heeger DJ (2011).
\href{https://doi.org/10.1038/nrn3136}{Normalization as a canonical
neural computation}. \emph{Nat Rev Neurosci} \textbf{13,} 51--62.

\bibitem[\citeproctext]{ref-Carandini1998}
Carandini M, Movshon JA \& Ferster D (1998).
\href{https://doi.org/10.1016/s0028-3908(98)00069-0}{Pattern adaptation
and cross-orientation interactions in the primary visual cortex}.
\emph{Neuropharmacology} \textbf{37,} 501--511.

\bibitem[\citeproctext]{ref-Chen2000}
Chen C, Foley JM \& Brainard DH (2000).
\href{https://doi.org/10.1016/s0042-6989(99)00227-8}{Detection of
chromoluminance patterns on chromoluminance pedestals i: Threshold
measurements}. \emph{Vision Res} \textbf{40,} 773--788.

\bibitem[\citeproctext]{ref-Dawson1954}
Dawson GD (1954). \href{https://doi.org/10.1016/0013-4694(54)90007-3}{A
summation technique for the detection of small evoked potentials}.
\emph{Electroencephalography and Clinical Neurophysiology} \textbf{6,}
65--84.

\bibitem[\citeproctext]{ref-De-Vries-Khoe1982}
De Vries-Khoe L \& Spekreijse H (1982). Maturation of luminance and
pattern EPs in man. \emph{Doc Ophthalmol Proc Ser} \textbf{31,}
461--475.

\bibitem[\citeproctext]{ref-Derrington1984}
Derrington AM, Krauskopf J \& Lennie P (1984).
\href{https://doi.org/10.1113/jphysiol.1984.sp015499}{Chromatic
mechanisms in lateral geniculate nucleus of macaque}. \emph{J Physiol}
\textbf{357,} 241--265.

\bibitem[\citeproctext]{ref-Di-Russo2007}
Di Russo F, Pitzalis S, Aprile T, Spitoni G, Patria F, Stella A,
Spinelli D \& Hillyard SA (2007).
\href{https://doi.org/10.1002/hbm.20276}{Spatiotemporal analysis of the
cortical sources of the steady-state visual evoked potential}. \emph{Hum
Brain Mapp} \textbf{28,} 323--334.

\bibitem[\citeproctext]{ref-Di-Russo2005}
Di Russo F, Pitzalis S, Spitoni G, Aprile T, Patria F, Spinelli D \&
Hillyard SA (2005).
\href{https://doi.org/10.1016/j.neuroimage.2004.09.029}{Identification
of the neural sources of the pattern-reversal VEP}. \emph{Neuroimage}
\textbf{24,} 874--886.

\bibitem[\citeproctext]{ref-Di-Russo2001}
Di Russo F, Spinelli D \& Morrone MC (2001).
\href{https://doi.org/10.1016/s0042-6989(01)00134-1}{Automatic gain
control contrast mechanisms are modulated by attention in humans:
Evidence from visual evoked potentials}. \emph{Vision Res} \textbf{41,}
2435--2447.

\bibitem[\citeproctext]{ref-Dobson1978}
Dobson V, Teller DY \& Belgum J (1978).
\href{https://doi.org/10.1016/0042-6989(78)90109-8}{Visual acuity in
human infants assessed with stationary stripes and phase-alternated
checkerboards}. \emph{Vision Res} \textbf{18,} 1233--1238.

\bibitem[\citeproctext]{ref-Du2023}
Du X, Liu L, Dong X \& Bao M (2023).
\href{https://doi.org/10.1111/nyas.14969}{Effects of altered-reality
training on interocular disinhibition in amblyopia}. \emph{Ann N Y Acad
Sci} \textbf{1522,} 126--138.

\bibitem[\citeproctext]{ref-Ellemberg1999}
Ellemberg D, Lewis TL, Liu CH \& Maurer D (1999).
\href{https://doi.org/10.1016/s0042-6989(98)00280-6}{Development of
spatial and temporal vision during childhood}. \emph{Vision Res}
\textbf{39,} 2325--2333.

\bibitem[\citeproctext]{ref-Foley1994}
Foley JM (1994). \href{https://doi.org/10.1364/josaa.11.001710}{Human
luminance pattern-vision mechanisms: Masking experiments require a new
model}. \emph{J Opt Soc Am A Opt Image Sci Vis} \textbf{11,} 1710--1719.

\bibitem[\citeproctext]{ref-Hamer1989}
Hamer RD, Norcia AM, Tyler CW \& Hsu-Winges C (1989).
\href{https://doi.org/10.1016/0042-6989(89)90004-7}{The development of
monocular and binocular VEP acuity}. \emph{Vision Res} \textbf{29,}
397--408.

\bibitem[\citeproctext]{ref-Harris1976}
Harris L, Atkinson J \& Braddick O (1976).
\href{https://doi.org/10.1038/264570a0}{Visual contrast sensitivity of a
6-month-old infant measured by the evoked potential}. \emph{Nature}
\textbf{264,} 570--571.

\bibitem[\citeproctext]{ref-Heeger1992}
Heeger DJ (1992).
\href{https://doi.org/10.1017/s0952523800009640}{Normalization of cell
responses in cat striate cortex}. \emph{Vis Neurosci} \textbf{9,}
181--197.

\bibitem[\citeproctext]{ref-Hoffmann2017}
Hoffmann MB, Brands J, Behrens-Baumann W \& Bach M (2017).
\href{https://doi.org/10.1007/s10633-017-9613-y}{VEP-based acuity
assessment in low vision}. \emph{Doc Ophthalmol} \textbf{135,} 209--218.

\bibitem[\citeproctext]{ref-Hu2023}
Hu J, Chen J, Ku Y \& Yu M (2023).
\href{https://doi.org/10.3389/fnins.2023.1280436}{Reduced interocular
suppression after inverse patching in anisometropic amblyopia}.
\emph{Front Neurosci} \textbf{17,} 1280436.

\bibitem[\citeproctext]{ref-Hubel1959}
Hubel DH \& Wiesel TN (1959).
\href{https://doi.org/10.1113/jphysiol.1959.sp006308}{Receptive fields
of single neurones in the cat's striate cortex}. \emph{J Physiol}
\textbf{148,} 574--591.

\bibitem[\citeproctext]{ref-Kuffler1957}
Kuffler SW, Fitzhugh R \& Barlow HB (1957).
\href{https://doi.org/10.1085/jgp.40.5.683}{Maintained activity in the
cat's retina in light and darkness}. \emph{J Gen Physiol} \textbf{40,}
683--702.

\bibitem[\citeproctext]{ref-Legge1979}
Legge GE (1979). \href{https://doi.org/10.1364/josa.69.000838}{Spatial
frequency masking in human vision: Binocular interactions}. \emph{J Opt
Soc Am} \textbf{69,} 838--847.

\bibitem[\citeproctext]{ref-Lennie1993}
Lennie P, Pokorny J \& Smith VC (1993).
\href{https://doi.org/10.1364/josaa.10.001283}{Luminance}. \emph{J Opt
Soc Am A} \textbf{10,} 1283--1293.

\bibitem[\citeproctext]{ref-Levitt2001}
Levitt JB, Schumer RA, Sherman SM, Spear PD \& Movshon JA (2001).
\href{https://doi.org/10.1152/jn.2001.85.5.2111}{Visual response
properties of neurons in the LGN of normally reared and visually
deprived macaque monkeys}. \emph{J Neurophysiol} \textbf{85,}
2111--2129.

\bibitem[\citeproctext]{ref-Lygo2021}
Lygo FA, Richard B, Wade AR, Morland AB \& Baker DH (2021).
\href{https://doi.org/10.1016/j.neuroimage.2021.117780}{Neural markers
of suppression in impaired binocular vision}. \emph{NeuroImage}
\textbf{230,} 117780.

\bibitem[\citeproctext]{ref-MacLennan2022}
MacLennan K, O'Brien S \& Tavassoli T (2022).
\href{https://doi.org/10.1007/s10803-021-05186-3}{In our own words: The
complex sensory experiences of autistic adults}. \emph{J Autism Dev
Disord} \textbf{52,} 3061--3075.

\bibitem[\citeproctext]{ref-MacLeod1979}
MacLeod DI \& Boynton RM (1979).
\href{https://doi.org/10.1364/josa.69.001183}{Chromaticity diagram
showing cone excitation by stimuli of equal luminance}. \emph{J Opt Soc
Am} \textbf{69,} 1183--1186.

\bibitem[\citeproctext]{ref-McKeefry1996}
McKeefry DJ, Russell MH, Murray IJ \& Kulikowski JJ (1996).
\href{https://doi.org/10.1017/s0952523800008543}{Amplitude and phase
variations of harmonic components in human achromatic and chromatic
visual evoked potentials}. \emph{Vis Neurosci} \textbf{13,} 639--653.

\bibitem[\citeproctext]{ref-Meese2009}
Meese TS \& Baker DH (2009).
\href{https://doi.org/10.1167/9.5.2}{Cross-orientation masking is speed
invariant between ocular pathways but speed dependent within them}.
\emph{J Vis} \textbf{9,} 2.1--15.

\bibitem[\citeproctext]{ref-Michelson1927}
Michelson A (1927). \emph{Studies in optics}. University of Chicago
Press.

\bibitem[\citeproctext]{ref-Morrone1986}
Morrone MC \& Burr DC (1986).
\href{https://doi.org/10.1038/321235a0}{Evidence for the existence and
development of visual inhibition in humans}. \emph{Nature} \textbf{321,}
235--237.

\bibitem[\citeproctext]{ref-Morrone1982}
Morrone MC, Burr DC \& Maffei L (1982).
\href{https://doi.org/10.1098/rspb.1982.0078}{Functional implications of
cross-orientation inhibition of cortical visual cells. I.
Neurophysiological evidence}. \emph{Proc R Soc Lond B Biol Sci}
\textbf{216,} 335--354.

\bibitem[\citeproctext]{ref-Movshon1975}
Movshon JA (1975).
\href{https://doi.org/10.1113/jphysiol.1975.sp011025}{The velocity
tuning of single units in cat striate cortex}. \emph{J Physiol}
\textbf{249,} 445--468.

\bibitem[\citeproctext]{ref-Norcia2015}
Norcia AM, Appelbaum LG, Ales JM, Cottereau BR \& Rossion B (2015).
\href{https://doi.org/10.1167/15.6.4}{The steady-state visual evoked
potential in vision research: {A} review}. \emph{Journal of vision}
\textbf{15,} 4--4.

\bibitem[\citeproctext]{ref-Norcia1985b}
Norcia AM \& Tyler CW (1985\emph{b}).
\href{https://doi.org/10.1016/0013-4694(85)91026-0}{Infant {VEP} acuity
measurements: Analysis of individual differences and measurement error}.
\emph{Electroencephalography and Clinical Neurophysiology} \textbf{61,}
359--369.

\bibitem[\citeproctext]{ref-Norcia1985a}
Norcia AM \& Tyler CW (1985\emph{a}).
\href{https://doi.org/10.1016/0042-6989(85)90217-2}{Spatial frequency
sweep {VEP}: Visual acuity during the first year of life}. \emph{Vision
Research} \textbf{25,} 1399--1408.

\bibitem[\citeproctext]{ref-Norcia1986}
Norcia AM, Tyler CW \& Allen D (1986).
\href{https://doi.org/10.1097/00006324-198601000-00003}{Electrophysiological
assessment of contrast sensitivity in human infants}. \emph{Am J Optom
Physiol Opt} \textbf{63,} 12--15.

\bibitem[\citeproctext]{ref-Norcia1990}
Norcia AM, Tyler CW \& Hamer RD (1990).
\href{https://doi.org/10.1016/0042-6989(90)90028-j}{Development of
contrast sensitivity in the human infant}. \emph{Vision Res}
\textbf{30,} 1475--1486.

\bibitem[\citeproctext]{ref-Pei2014}
Pei F, Baldassi S \& Norcia AM (2014). Electrophysiological measures of
low-level vision reveal spatial processing deficits and hemispheric
asymmetry in autism spectrum disorder. \emph{J Vis}; DOI:
\href{https://doi.org/10.1167/14.11.3}{10.1167/14.11.3}.

\bibitem[\citeproctext]{ref-Pei2017}
Pei F, Baldassi S, Tsai JJ, Gerhard HE \& Norcia AM (2017).
\href{https://doi.org/10.1016/j.visres.2016.03.010}{Development of
contrast normalization mechanisms during childhood and adolescence}.
\emph{Vision Res} \textbf{133,} 12--20.

\bibitem[\citeproctext]{ref-Peirce2007}
Peirce JW (2007). \href{https://doi.org/10.1167/7.6.13}{The potential
importance of saturating and supersaturating contrast response functions
in visual cortex}. \emph{J Vis} \textbf{7,} 13.

\bibitem[\citeproctext]{ref-Petrov2005}
Petrov Y, Carandini M \& McKee S (2005).
\href{https://doi.org/10.1523/JNEUROSCI.2871-05.2005}{Two distinct
mechanisms of suppression in human vision}. \emph{J Neurosci}
\textbf{25,} 8704--8707.

\bibitem[\citeproctext]{ref-Polat1993}
Polat U \& Sagi D (1993).
\href{https://doi.org/10.1016/0042-6989(93)90081-7}{Lateral interactions
between spatial channels: Suppression and facilitation revealed by
lateral masking experiments}. \emph{Vision Res} \textbf{33,} 993--999.

\bibitem[\citeproctext]{ref-Porciatti2000}
Porciatti V, Bonanni P, Fiorentini A \& Guerrini R (2000).
\href{https://doi.org/10.1038/72972}{Lack of cortical contrast gain
control in human photosensitive epilepsy}. \emph{Nature Neuroscience}
\textbf{3,} 259--263.

\bibitem[\citeproctext]{ref-Regan1966}
Regan D (1966). \href{https://doi.org/10.1016/0013-4694(66)90088-5}{Some
characteristics of average steady-state and transient responses evoked
by modulated light}. \emph{Electroencephalogr Clin Neurophysiol}
\textbf{20,} 238--248.

\bibitem[\citeproctext]{ref-Regan1988}
Regan MP \& Regan D (1988).
\href{https://doi.org/10.1016/S0022-5193(88)80323-0}{A frequency domain
technique for characterizing nonlinearities in biological systems}.
\emph{Journal of Theoretical Biology} \textbf{133,} 293--317.

\bibitem[\citeproctext]{ref-Ridder2004}
Ridder WH 3rd (2004).
\href{https://doi.org/10.1007/s10633-004-8053-7}{Methods of visual
acuity determination with the spatial frequency sweep visual evoked
potential}. \emph{Doc Ophthalmol} \textbf{109,} 239--247.

\bibitem[\citeproctext]{ref-Shapley1980}
Shapley RM \& Victor JD (1980).
\href{https://doi.org/10.1113/jphysiol.1980.sp013259}{The effect of
contrast on the non-linear response of the y cell}. \emph{J Physiol}
\textbf{302,} 535--547.

\bibitem[\citeproctext]{ref-Shibata2008}
Shibata K, Yamane K, Otuka K \& Iwata M (2008).
\href{https://doi.org/10.1016/j.jns.2008.04.004}{Abnormal visual
processing in migraine with aura: A study of steady-state visual evoked
potentials}. \emph{J Neurol Sci} \textbf{271,} 119--126.

\bibitem[\citeproctext]{ref-Simmons2009}
Simmons DR, Robertson AE, McKay LS, Toal E, McAleer P \& Pollick FE
(2009). \href{https://doi.org/10.1016/j.visres.2009.08.005}{Vision in
autism spectrum disorders}. \emph{Vision Res} \textbf{49,} 2705--2739.

\bibitem[\citeproctext]{ref-Skoczenski1998}
Skoczenski AM \& Norcia AM (1998).
\href{https://doi.org/10.1038/35630}{Neural noise limitations on infant
visual sensitivity}. \emph{Nature} \textbf{391,} 697--700.

\bibitem[\citeproctext]{ref-Tadin2003}
Tadin D, Lappin JS, Gilroy LA \& Blake R (2003).
\href{https://doi.org/10.1038/nature01800}{Perceptual consequences of
centre-surround antagonism in visual motion processing}. \emph{Nature}
\textbf{424,} 312--315.

\bibitem[\citeproctext]{ref-Tolhurst1981}
Tolhurst DJ, Movshon JA \& Thompson ID (1981).
\href{https://doi.org/10.1007/BF00238900}{The dependence of response
amplitude and variance of cat visual cortical neurones on stimulus
contrast}. \emph{Exp Brain Res} \textbf{41,} 414--419.

\bibitem[\citeproctext]{ref-Tsai2011}
Tsai JJ, Norcia AM, Ales JM \& Wade AR (2011).
\href{https://doi.org/10.1002/ana.22462}{Contrast gain control
abnormalities in idiopathic generalized epilepsy}. \emph{Annals of
Neurology} \textbf{70,} 574--582.

\bibitem[\citeproctext]{ref-Tsai2012}
Tsai JJ, Wade AR \& Norcia AM (2012).
\href{https://doi.org/10.1523/JNEUROSCI.4485-11.2012}{Dynamics of
normalization underlying masking in human visual cortex}. \emph{The
Journal of Neuroscience: The Official Journal of the Society for
Neuroscience} \textbf{32,} 2783--2789.

\bibitem[\citeproctext]{ref-Tyler1985}
Tyler CW \& Apkarian PA (1985).
\href{https://doi.org/10.1016/0042-6989(85)90183-x}{Effects of contrast,
orientation and binocularity in the pattern evoked potential}.
\emph{Vision Res} \textbf{25,} 755--766.

\bibitem[\citeproctext]{ref-Tyler1979}
Tyler CW, Apkarian P, Levi DM \& Nakayama K (1979).
\href{https://www.ncbi.nlm.nih.gov/pubmed/447469}{Rapid assessment of
visual function: An electronic sweep technique for the pattern visual
evoked potential}. \emph{Investigative Ophthalmology \& Visual Science}
\textbf{18,} 703--713.

\bibitem[\citeproctext]{ref-Vilidaite2018}
Vilidaite G, Norcia AM, West RJH, Elliott CJH, Pei F, Wade AR \& Baker
DH (2018). \href{https://doi.org/10.1098/rspb.2018.2255}{Autism sensory
dysfunction in an evolutionarily conserved system}. \emph{Proceedings
Biological Sciences} \textbf{285,} 20182255.

\bibitem[\citeproctext]{ref-Zheng2019}
Zheng X, Xu G, Zhi Y, Wang Y, Han C, Wang B, Zhang S, Zhang K \& Liang R
(2019). \href{https://doi.org/10.1016/j.visres.2019.07.003}{Objective
and quantitative assessment of interocular suppression in strabismic
amblyopia based on steady-state motion visual evoked potentials}.
\emph{Vision Res} \textbf{164,} 44--52.

\end{CSLReferences}



\end{document}
